%---------------------------------------------------------------------------------------------------
\begin{block}{3. Datasets}
%---------------------------------------------------------------------------------------------------
% - meter imagens representativas em vez de texto

In \cite{Anagnostopoulos2021} the authors show that this model of $f(Q)$ is statistically equivalent to $\Lambda$CDM at low redshifts. Here, we constrain this model using both low and high redshift observables, namely \gls{SnIa}, \gls{CMB} and \gls{SS} events.

Due to the lack of measurements of \gls{SS} events, we generate mock catalogs assuming $\Lambda$CDM, to asses the constraining power of the \gls{LIGO}, the \gls{LISA} and the \gls{ET}.

% begin columns
\vspace{-0.25cm}
\begin{columns}[T,totalwidth=\colwidth]

% begin left column
\begin{column}{0.475\colwidth}
    \begin{exampleblock}{}
        \heading{Standard Sirens}
        \begin{figure}[h!]
            \includegraphics[width=0.475\colwidth,center]{figures/SS.jpg}
            \scriptsize{Image credits: IceCube Collaboration (Modified)}
        \end{figure}
    \end{exampleblock}
\end{column}

% begin right column
\begin{column}{0.475\colwidth}
    \begin{exampleblock}{}
        \heading{Type Ia Supernova}
        \begin{figure}[h!]
            \includegraphics[width=0.475\colwidth,center]{figures/SnIa.png}
        \end{figure}
    \end{exampleblock}
\end{column}

\end{columns}

% lower column spanning entire block
\begin{exampleblock}{}
    \heading{Cosmic Microwave Background}
    \begin{figure}[h!]
        \includegraphics[width=0.475\colwidth,center]{figures/CMB.jpg}
        \scriptsize{Image credits: ESA and the Planck Collaboration}
    \end{figure}
\end{exampleblock}

\end{block}