%---------------------------------------------------------------------------------------------------
\begin{block}{2. Dynamical Systems Analysis}
%---------------------------------------------------------------------------------------------------
% -

We employ a dynamical systems analysis to assess the ability of this model of $f(Q)$ to replicate the various distinct phases that the universe has undergone. To do so we define the quantities

\vspace{-1cm}
\begin{align}
    x_1 &\equiv \frac{\Omega_m}{E^2 e^{\lambda / E^2} a^3} \,,
    &
    x_2 &\equiv \frac{\Omega_r}{E^2 e^{\lambda / E^2} a^4} \,,
\end{align}

where $x_1$ is related to the evolution of the matter density in the universe and $x_2$ with the radiation density. The trajectories and fixed points in phase space are presented in the figure below.

\begin{figure}[h!]
    \centering
    \includegraphics[width=0.95\columnwidth]{figures/dynamical-system.png}
\end{figure}

From the previous figure we can see that \textbf{only the trajectories in the light blue region}, corresponding to $\lambda > 0$, \textbf{replicate the dynamics of the $\boldsymbol{\Lambda}$CDM universe}. The universe starts in a radiation dominated epoch, moves towards matter domination, and ends up in a $\lambda$ dominated regime, corresponding to an universe in accelerated expansion.

As for the brown region, corresponding to $\lambda = 0$, we have the dynamics of a CDM universe, and in the dark blue region, corresponding to $\lambda < 0$, we have a universe with a big crunch.

% Using the previous analysis and evaluating the modified Friedmann equation at the present day, we can express the value of $\lambda$ as a function of the energy-matter content of the universe as

% \begin{equation}
%     \lambda = \frac{1}{2} + W_0\left( -\frac{\Omega_m + \Omega_r}{2e^{1/2}} \right) \,,
% \end{equation}

% where $W_0$ is the main branch of the Lambert function.

\end{block}