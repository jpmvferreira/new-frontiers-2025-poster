%---------------------------------------------------------------------------------------------------
\begin{block}{4. Constraints}
%---------------------------------------------------------------------------------------------------
% -

\begin{figure}[h!]
    \centering
    \begin{subfigure}[b]{0.49\textwidth}
        \centering
        \heading{$\boldsymbol{\Lambda}$CDM}
        \includegraphics[width=\textwidth]{figures/LCDM-ET-LISA-CMB-SnIa.png}
     \end{subfigure}
     \hfill
     \begin{subfigure}[b]{0.49\textwidth}
        \centering
        \heading{$\boldsymbol{f(Q) = Qe^{\lambda Q_0/Q}}$}
        \includegraphics[width=\textwidth]{figures/fotis-ET-LISA-CMB-SnIa.png}
     \end{subfigure}
     %\caption{The constraints set by the forecast \gls{SS} events for \gls{LISA} and the \gls{ET}, by \gls{SnIa} and by the \gls{CMB} on $\Lambda$CDM (on the left) versus this model of $f(Q)$ (on the right). Neither of the tensions that exists in this model of $f(Q)$ appears when assuming a $\Lambda$CDM universe.}
     %\label{fig:constraints}
\end{figure}

In the previous plots, we can see the constraints set by each of the datasets used in this work for $\Lambda$CDM, on the left, and our model of $f(Q)$, in the right. We define the dimensionless Hubble constant as $h = H_0/100 \,\, \text{s Mpc km}^{-1}$.

\textbf{We see a tension in $\boldsymbol{\Omega_m}$ between \gls{SnIa} and the \gls{CMB}}. This happens because even though this model falls back to \gls{GR} at high redshifts, the values of the abundances are modified when compared to $\Lambda$CDM, indirectly modifying the behavior of the universe at early times.

\textbf{Using \gls{SS} events forecast for \gls{LISA} and the \gls{ET}, a tension in the value of $\boldsymbol{h}$ arises}, making future gravitational wave observatories prime candidates for testing this and similar models.

\gls{LIGO} is not expected to be provide meaningful constraints due to its lower accuracy.

\end{block}