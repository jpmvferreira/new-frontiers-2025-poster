%---------------------------------------------------------------------------------------------------
\begin{block}{1. $\boldsymbol{f(Q)}$ Cosmology}
%---------------------------------------------------------------------------------------------------
% -

Despite the successes of \gls{GR}, which ascribes gravity to the curvature of spacetime, its geometry might not necessarily be due to curvature but due to other geometrical objects, such as non-metricity or torsion.

\vspace{0.25cm}
\heading{\hspace{0.5cm} Curvature \hspace{5.5cm} Torsion \hspace{4.75cm} Non-metricity}
\begin{overpic}
    [width=0.95\columnwidth,center]{figures/RTQ.pdf}
    \put(100,2.5){\rotatebox{90}{\scriptsize{Image credits: arXiv:2106.13793}}}
\end{overpic}
\vspace{0cm}  % for some reason, this adds white space...

We study a specific model of $f(Q)$ gravity, where the Ricci scalar in the action is replaced by an arbitrary function of the non-metricity scalar, that aims to replicate the late-time expansion of the universe. It does so by departing from \gls{GR} via an extra multiplicative term, given by \cite{Anagnostopoulos2021}

\begin{equation}
    f(Q) = Q e^{\lambda Q_0/Q} \,.
\end{equation}

We consider a \textbf{flat, homogeneous and isotropic universe}, described by a flat \gls{FLRW} metric, and permeated by a perfect fluid \textbf{composed solely of matter and radiation, without dark energy}. Under the previous assumptions the first Friedmann equation is modified and reads

\begin{equation}
    \label{eq:modified-friedmann}
    (E^2(z) - 2 \lambda)e^{\lambda/E^2(z)} = \Omega_m(1+z)^3 + \Omega_r(1+z)^4 \,,
\end{equation}

where $E(z) \equiv H(z)/H_0$, with $H(z)$ the Hubble function. 

From the previous equation we can see that this model modifies the universe at late time $(E(z) \ll 1)$ while recovering \gls{GR} at high redshifts $(E(z) \gg 1)$, with modified matter and radiation abundances. 

This type of models also modify the cosmology at the perturbative level \cite{Belgacem2017a}.

\end{block}