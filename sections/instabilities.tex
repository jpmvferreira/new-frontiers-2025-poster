%---------------------------------------------------------------------------------------------------
\begin{block}{Instability Timescales}
%---------------------------------------------------------------------------------------------------
% - TODO

Since all solutions are unstable, \textbf{we characterize instability timescales by defining $\tau_i$ and $\tau_f$} for each scenario. For \textit{Fission}, $\tau_i$ ($\tau_f$) marks the instant of time when the \gls{BH} crosses the radius enclosing 1\% (99\%) of the cloud's total charge; for \textit{Absorption}, $\tau_i$ ($\tau_f$) correspond to the instant of time when the cloud has 99\% (1\%) of its initial charge.

\begin{figure}[h!]
    \centering
    \begin{minipage}[b]{0.49\textwidth}
        \centering
        {\large \Raleway \textmd{Fission}} \\
        \vspace{0.75em}
        \includegraphics[width=\textwidth]{figures/timescales/fission.jpg}
    \end{minipage}
    \hfill
    \begin{minipage}[b]{0.49\textwidth}
        \centering
        {\large \Raleway \textmd{Absorption}} \\
        \vspace{0.75em}
        \includegraphics[width=\textwidth]{figures/timescales/absorption.jpg}
    \end{minipage}
    \caption{Definition of $\tau_i$ and $\tau_f$ for \textit{Fission} scenario (left) and \textit{Absorption} scenario (right).}
\end{figure}

In the figures below we can see the timescales for the \textit{Fission} scenario. We can see that $\tau_i$ does not vary with $\omega$ for solutions in the upper branch and for solutions in the lower branch it decreases with $\omega$. As for $\Delta \tau$, it increases with $\omega$ in the lower branch and has a non-monotonic pattern in the upper branch.

\begin{figure}[h!]
    \centering
    \begin{minipage}[b]{0.49\textwidth}
        \centering
        \includegraphics[width=\textwidth]{figures/plots/fission_taui.png}
    \end{minipage}
    \hfill
    \begin{minipage}[b]{0.49\textwidth}
        \centering
        \includegraphics[width=\textwidth]{figures/plots/fission_deltatau.png}
    \end{minipage}
    \caption{$\tau_i$ (left) and $\Delta \tau$ (right) vs $\omega$ for $r_h = 0.2$ in the \textit{Fission} scenario.}
\end{figure}

The \textit{Absorption} only takes place in solutions which have more compact hair, located in the lower branch. By fixing $\omega$ and varying $r_h$, we can see that both $\tau_i$ and $\Delta \tau$ decrease with $r_h$.

\begin{figure}[h!]
    \centering
    \begin{minipage}[b]{0.49\textwidth}
        \centering
        \includegraphics[width=\textwidth]{figures/plots/absorption_taui.png}
    \end{minipage}
    \hfill
    \begin{minipage}[b]{0.49\textwidth}
        \centering
        \includegraphics[width=\textwidth]{figures/plots/absorption_deltatau.png}
    \end{minipage}
    \caption{$\tau_i$ (left) and $\Delta \tau$ (right) vs $r_h$ for $\omega = 0.95$ in the \textit{Absorption} scenario.}
\end{figure}

\vspace{-1em}  % next block is too far away

\end{block}
