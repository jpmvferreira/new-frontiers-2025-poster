% TEMP:
% - https://github.com/andiac/gemini-cam
% - a fork of https://github.com/anishathalye/gemini
% - also refer to https://github.com/k4rtik/uchicago-poster

% TO-DO/NOTAS: 
% - (!) abstract pode ser um bloco com fundo diferente ou mesmo com uma ornamentacao mais pipi para se destacar.
% - orientar o tema/color scheme para ter esta template organizada (l8er que parece tramado)
% - footer estar ligeiramente offset, espaco a mais a esquerda e a direita e eu n sei pq...
% - logos tb estao ligeiramente offset, acho que tem valores hardcoded na template
% - o alert block eh um bloco sem header so para o caso de querer continuar um anterior

%---------------------------------------------------------------------------------------------------
% Preamble
%---------------------------------------------------------------------------------------------------
\documentclass[final,20pt]{beamer}

% import 
\usepackage[
orientation=portrait,
size=a1,
scale=1.15  % 20 pt * 1.15 = 23pt
]{beamerposter}

% for N columns pick \sepwidth and \colwidth such that (N+1)*\sepwidth + N*\colwidth = \paperwidth
\newlength{\sepwidth}
\newlength{\colwidth}
\setlength{\sepwidth}{0.025\paperwidth}
\setlength{\colwidth}{0.4625\paperwidth}
\newcommand{\separatorcolumn}{\begin{column}{\sepwidth}\end{column}}

% import the theme and the color theme
\makeatletter
\def\input@path{{styles/}}
\makeatother
\usetheme{gemini}     
\usecolortheme{nott}  

% figures & tables
\usepackage{graphicx}
\usepackage[export]{adjustbox}
\usepackage{caption}
\usepackage{subcaption}
\usepackage{booktabs}

% math related packages
\usepackage{amsmath}
\usepackage{amssymb} 
\usepackage{amsthm}

% import and setup glossary
\usepackage{glossaries-extra}
%\makeglossaries  % helper script to print glossary
\glsdisablehyper  % disable hyperlinks to glossary
\newglossaryentry{BH}{
    name=BH,
    description=Black Hole,
    first=Black Hole (BH),
    plural=BHs,
    descriptionplural=Black Holes,
    firstplural=Black Holes (BHs)
}

\newglossaryentry{BS}{
    name=BS,
    description=Boson Star,
    first=Boson Star (BS),
    plural=BSs,
    descriptionplural=Boson Stars,
    firstplural=Boson Stars (BSs)
}

\newglossaryentry{EM}{
    name=EM,
    description=electromagnetic,
    first=electromagnetic (EM)
}

\newglossaryentry{GR}{
    name=GR,
    description=General Relativity,
    first=General Relativity (GR)
}

\newglossaryentry{EoM}{
    name=EoM,
    description=equation of motion,
    first=equation of motion (EoM),
    plural=EoM,
    descriptionplural=equations of motion,
    firstplural=equations of motion (EoM)
}


% import and configure bibLaTeX, a more modern alternative to natbib
\usepackage[
backend=biber,  % more modern backend compared to bibtex
style=numeric,  % citation style
sorting=none,   % all entries are processed in citation order
maxbibnames=4,  % maximum number of authors before truncating
maxcitenames=1  % number of authors to show before the "et al."
]{biblatex}
\addbibresource{bibliography.bib}

% decrease font size for bibliography
\renewcommand*{\bibfont}{\normalfont\footnotesize}

% suppress some fields in the bibliography to save space
\AtEveryBibitem{\clearfield{eprint}}
\AtEveryBibitem{\clearfield{eprintclass}}
\AtEveryBibitem{\clearfield{eprint}}
\AtEveryBibitem{\clearfield{eprinttype}}
\AtEveryBibitem{\clearfield{doi}}

% configure hyperref (already loaded by default)
\hypersetup{
	colorlinks=true,    % color links
	linkcolor=blue,     % set color of internal links
	citecolor=green,    % set color of links to bibliography
	filecolor=magenta,  % set color of file links
	urlcolor=cyan       % set color of urls      
}

% make arguments bold
\usepackage{bm}

% title
\title{Testing {\boldmath $\Lambda$}-Free {\boldmath $f(Q)$} Cosmology}

% author(s)
% using hacky workaround for affiliations (see: https://github.com/anishathalye/gemini/issues/32)
\author{
\underline{\textbf{José Ferreira}} $^\text{\ttfamily 1,2}$ \and 
Tiago Barreiro $^\text{\ttfamily 1,3}$ \and
José Mimoso $^\text{\ttfamily 1,2}$ \and 
Nelson J. Nunes $^\text{\ttfamily 1,2}$
}

% affiliation(s)
\institute[shortinst]{
\inst{1} \vspace{2.5mm} Institute of Astrophysics and Space Sciences
\samelineand 
\inst{2} \vspace{2.5mm} Faculty of Sciences of the University of Lisbon
\and
\vspace{-10mm}
\inst{3} \vspace{2.5mm} ECEO, Universidade Lusófona de Humanidades e Tecnologia
}

% extra information to display below affiliations
\extratitle{\underline{\textbf{Based on arXiv:XXXXXX}}}

% footer
% (comment out for no footer)
\footercontent{
    \vspace{1.5ex}
    \hspace{\sepwidth}
    \begin{minipage}{0.2\paperwidth}
        \raggedright
        jpmferreira@ciencias.ulisboa.pt
    \end{minipage}
    \hfill
    \begin{minipage}{0.2\paperwidth}
        \centering
        \textbf{Testing {\boldmath $\Lambda$}-Free {\boldmath $f(Q)$} Cosmology}
    \end{minipage}
    \hfill
    \begin{minipage}{0.2\paperwidth}
        \raggedleft
        José Ferreira
    \end{minipage}
    \hspace{\sepwidth}
    \vspace{1.5ex}
}

% logo(s)
% (comment out for none)
\logoright{\includegraphics[height=4.77cm,width=11.75cm]{figures/logos/FCUL.png}}
\logoleft{\includegraphics[height=4.77cm,width=11.75cm]{figures/logos/IA.png}}

% text on top of imagesx
\usepackage{overpic}

%---------------------------------------------------------------------------------------------------
% Begin of Document
%---------------------------------------------------------------------------------------------------
\begin{document}

\begin{frame}[t]
\begin{columns}[t]
\separatorcolumn

% begin left column
\begin{column}{\colwidth}
    %---------------------------------------------------------------------------------------------------
\begin{block}{Abstract}
%---------------------------------------------------------------------------------------------------
% -

\textbf{We study a cosmological model of $\boldsymbol{f(Q)}$ gravity that accounts for the current accelerated expansion of the universe, without dark energy or extra free parameters}. We use dynamical system techniques to study this model and provide constraints using \gls{SnIa}, \gls{CMB} data and forecast \gls{SS} events.
  
\end{block}

    %---------------------------------------------------------------------------------------------------
\begin{block}{1. $\boldsymbol{f(Q)}$ Cosmology}
%---------------------------------------------------------------------------------------------------
% -

Despite the successes of \gls{GR}, which ascribes gravity to the curvature of spacetime, its geometry might not necessarily be due to curvature but due to other geometrical objects, such as non-metricity or torsion.

\vspace{0.25cm}
\heading{\hspace{0.5cm} Curvature \hspace{5.5cm} Torsion \hspace{4.75cm} Non-metricity}
\begin{overpic}
    [width=0.95\columnwidth,center]{figures/RTQ.pdf}
    \put(100,2.5){\rotatebox{90}{\scriptsize{Image credits: arXiv:2106.13793}}}
\end{overpic}
\vspace{0cm}  % for some reason, this adds white space...

We study a specific model of $f(Q)$ gravity, where the Ricci scalar in the action is replaced by an arbitrary function of the non-metricity scalar, that aims to replicate the late-time expansion of the universe. It does so by departing from \gls{GR} via an extra multiplicative term, given by \cite{Anagnostopoulos2021}

\begin{equation}
    f(Q) = Q e^{\lambda Q_0/Q} \,.
\end{equation}

We consider a \textbf{flat, homogeneous and isotropic universe}, described by a flat \gls{FLRW} metric, and permeated by a perfect fluid \textbf{composed solely of matter and radiation, without dark energy}. Under the previous assumptions the first Friedmann equation is modified and reads

\begin{equation}
    \label{eq:modified-friedmann}
    (E^2(z) - 2 \lambda)e^{\lambda/E^2(z)} = \Omega_m(1+z)^3 + \Omega_r(1+z)^4 \,,
\end{equation}

where $E(z) \equiv H(z)/H_0$, with $H(z)$ the Hubble function. 

From the previous equation we can see that this model modifies the universe at late time $(E(z) \ll 1)$ while recovering \gls{GR} at high redshifts $(E(z) \gg 1)$, with modified matter and radiation abundances. 

This type of models also modify the cosmology at the perturbative level \cite{Belgacem2017a}.

\end{block}
    %---------------------------------------------------------------------------------------------------
\begin{block}{2. Dynamical Systems Analysis}
%---------------------------------------------------------------------------------------------------
% -

We employ a dynamical systems analysis to assess the ability of this model of $f(Q)$ to replicate the various distinct phases that the universe has undergone. To do so we define the quantities

\vspace{-1cm}
\begin{align}
    x_1 &\equiv \frac{\Omega_m}{E^2 e^{\lambda / E^2} a^3} \,,
    &
    x_2 &\equiv \frac{\Omega_r}{E^2 e^{\lambda / E^2} a^4} \,,
\end{align}

where $x_1$ is related to the evolution of the matter density in the universe and $x_2$ with the radiation density. The trajectories and fixed points in phase space are presented in the figure below.

\begin{figure}[h!]
    \centering
    \includegraphics[width=0.95\columnwidth]{figures/dynamical-system.png}
\end{figure}

From the previous figure we can see that \textbf{only the trajectories in the light blue region}, corresponding to $\lambda > 0$, \textbf{replicate the dynamics of the $\boldsymbol{\Lambda}$CDM universe}. The universe starts in a radiation dominated epoch, moves towards matter domination, and ends up in a $\lambda$ dominated regime, corresponding to an universe in accelerated expansion.

As for the brown region, corresponding to $\lambda = 0$, we have the dynamics of a CDM universe, and in the dark blue region, corresponding to $\lambda < 0$, we have a universe with a big crunch.

% Using the previous analysis and evaluating the modified Friedmann equation at the present day, we can express the value of $\lambda$ as a function of the energy-matter content of the universe as

% \begin{equation}
%     \lambda = \frac{1}{2} + W_0\left( -\frac{\Omega_m + \Omega_r}{2e^{1/2}} \right) \,,
% \end{equation}

% where $W_0$ is the main branch of the Lambert function.

\end{block}
\end{column}

\separatorcolumn

% begin right column
\begin{column}{\colwidth}
    %---------------------------------------------------------------------------------------------------
\begin{block}{3. Datasets}
%---------------------------------------------------------------------------------------------------
% - meter imagens representativas em vez de texto

In \cite{Anagnostopoulos2021} the authors show that this model of $f(Q)$ is statistically equivalent to $\Lambda$CDM at low redshifts. Here, we constrain this model using both low and high redshift observables, namely \gls{SnIa}, \gls{CMB} and \gls{SS} events.

Due to the lack of measurements of \gls{SS} events, we generate mock catalogs assuming $\Lambda$CDM, to asses the constraining power of the \gls{LIGO}, the \gls{LISA} and the \gls{ET}.

% begin columns
\vspace{-0.25cm}
\begin{columns}[T,totalwidth=\colwidth]

% begin left column
\begin{column}{0.475\colwidth}
    \begin{exampleblock}{}
        \heading{Standard Sirens}
        \begin{figure}[h!]
            \includegraphics[width=0.475\colwidth,center]{figures/SS.jpg}
            \scriptsize{Image credits: IceCube Collaboration (Modified)}
        \end{figure}
    \end{exampleblock}
\end{column}

% begin right column
\begin{column}{0.475\colwidth}
    \begin{exampleblock}{}
        \heading{Type Ia Supernova}
        \begin{figure}[h!]
            \includegraphics[width=0.475\colwidth,center]{figures/SnIa.png}
        \end{figure}
    \end{exampleblock}
\end{column}

\end{columns}

% lower column spanning entire block
\begin{exampleblock}{}
    \heading{Cosmic Microwave Background}
    \begin{figure}[h!]
        \includegraphics[width=0.475\colwidth,center]{figures/CMB.jpg}
        \scriptsize{Image credits: ESA and the Planck Collaboration}
    \end{figure}
\end{exampleblock}

\end{block}
    %---------------------------------------------------------------------------------------------------
\begin{block}{4. Constraints}
%---------------------------------------------------------------------------------------------------
% -

\begin{figure}[h!]
    \centering
    \begin{subfigure}[b]{0.49\textwidth}
        \centering
        \heading{$\boldsymbol{\Lambda}$CDM}
        \includegraphics[width=\textwidth]{figures/LCDM-ET-LISA-CMB-SnIa.png}
     \end{subfigure}
     \hfill
     \begin{subfigure}[b]{0.49\textwidth}
        \centering
        \heading{$\boldsymbol{f(Q) = Qe^{\lambda Q_0/Q}}$}
        \includegraphics[width=\textwidth]{figures/fotis-ET-LISA-CMB-SnIa.png}
     \end{subfigure}
     %\caption{The constraints set by the forecast \gls{SS} events for \gls{LISA} and the \gls{ET}, by \gls{SnIa} and by the \gls{CMB} on $\Lambda$CDM (on the left) versus this model of $f(Q)$ (on the right). Neither of the tensions that exists in this model of $f(Q)$ appears when assuming a $\Lambda$CDM universe.}
     %\label{fig:constraints}
\end{figure}

In the previous plots, we can see the constraints set by each of the datasets used in this work for $\Lambda$CDM, on the left, and our model of $f(Q)$, in the right. We define the dimensionless Hubble constant as $h = H_0/100 \,\, \text{s Mpc km}^{-1}$.

\textbf{We see a tension in $\boldsymbol{\Omega_m}$ between \gls{SnIa} and the \gls{CMB}}. This happens because even though this model falls back to \gls{GR} at high redshifts, the values of the abundances are modified when compared to $\Lambda$CDM, indirectly modifying the behavior of the universe at early times.

\textbf{Using \gls{SS} events forecast for \gls{LISA} and the \gls{ET}, a tension in the value of $\boldsymbol{h}$ arises}, making future gravitational wave observatories prime candidates for testing this and similar models.

\gls{LIGO} is not expected to be provide meaningful constraints due to its lower accuracy.

\end{block}
    %---------------------------------------------------------------------------------------------------
\begin{block}{5. Conclusions}
%---------------------------------------------------------------------------------------------------
% -

We conclude that \textbf{although this model of $\boldsymbol{f(Q)}$ gravity is able to account for the accelerated expansion of the universe at late times, without requiring dark energy, it shows a tension between high and low redshift measurements}.

We expect that future gravitational wave observatories will be able to provide us with data that can be used to rule this model out, without relying on high redshift measurements.

\end{block}
    \begin{block}{References}
    \printbibliography
\end{block}
\end{column}

\separatorcolumn
\end{columns}

% hacky acknowledgments
% i have to find a way to place this on top of the footer systematically...
\vspace{-0.4cm}
\begin{center}
    \scriptsize{
    The authors would like to acknowledge financial support from Fundação Para a Ciência e a Tecnologia via the following projects: EXPL/FIS-AST/1368/2021, PTDC/FIS-AST/0054/2021, UIDB/04434/2020, UIDP/04434/2020 and CERN/FIS-PAR/0037/2019.
    }
\end{center}

\end{frame}

\end{document}
